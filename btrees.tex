\chapter{B-Trees}
\label{chap:btrees}

% stolen from CLRS
B-trees are balanced search trees designed to work well on disks or other direct access secondary storage devices. B-trees are similar to red-black trees, but they are better at minimizing disk I/O operations. Many database systems use B-trees, or variants of B-trees, to store information.

B-trees differ from red-black trees in that B-tree nodes may have many children,
from a few to thousands. That is, the ``branching factor'' of a B-tree can be quite
large, although it usually depends on characteristics of the disk unit used. B-trees
are similar to red-black trees in that every $n$-node B-tree has height $O\left(\log n\right)$. The
exact height of a B-tree can be considerably less than that of a red-black tree,
however, because its branching factor, and hence the base of the logarithm that
expresses its height, can be much larger. Therefore, we can also use B-trees to
implement many dynamic-set operations in time $O\left(\log n\right)$.

\section{Definition of B-trees}

\section{Basic operations}

\section{Deleting a key}